%%%%%%%%%%%%%%%%%%%%%%%%%%%%%%%%%%%%%%%%%%%%%%%%%%%%%%%%%%%%%%%%%%%%%%%%%%
%%%%%%%%%%%%   CAPTER 1   %%%%%%%%%%%%%%%%%%%%%%%%%%%%%%%%%%%%%%%%%%%%%%%%
%%%%%%%%%%%%%%%%%%%%%%%%%%%%%%%%%%%%%%%%%%%%%%%%%%%%%%%%%%%%%%%%%%%%%%%%%%
\chapter{Introduction}
\label{chap:introduction}
The use of wireless communication is on a steady rise. Not only in the home and industry environment,
but also in medical facilities. There are wireless patient monitoring systems available, as well as 
endoscopes and other surgical equipment, without the need for a cable.
Using wireless equipment reduces the struggle with cable management in the surgery and enables the operating
personal to be even more effective.\\
Another use of wireless equipment is post surgery/disease treatment in rural areas. Deploying a way to communicate
with the attending physician, without the need to take a long ride to the next hospital or to get to the patient
eases the treatment a lot. On side equipment can be installed at the patient side for easy communication and monitoring.\\
In this work we will focus on the second use case\\
For both use cases a robust and reliable wireless communication is needed.
In the case of the hospital a lot of moving people, running equipment and walls can attenuate the signal and distort wave
propagation. In the case of home installed equipment we expect more static attenuation, due to walls and running equipment, like micro wave ovens
and TVs. Another muting effect that we expect is, that wireless routers are usually hidden somewhere, which causes sub optimal wave propagation.\\

Resource allocation algorithms aim to provide high and robust throughput. As there are multiple algorithms available, which we expect to react differently on
disturbing effects an evaluation is needed to choose the best for a given application.

%%%%%%%%%%%%%%%%%%%%%%%%%%%%%%%%%%%%%
%%%%%%%%%%%%%%%%%%%%%%%%%%%%%%%%%%%%%
%%%%%%%%%%%%   SECTION   %%%%%%%%%%%%
%%%%%%%%%%%%%%%%%%%%%%%%%%%%%%%%%%%%%
%%%%%%%%%%%%%%%%%%%%%%%%%%%%%%%%%%%%%
\section{Problem Statement}
\label{sec:intro:probstatement}

The problem space is definced by the following.

%%%%%%%%%%%%%%%%%%%%%%%%%%%%%%%%%%%%%
%%%%%%%%%%%%%%%%%%%%%%%%%%%%%%%%%%%%%
%%%%%%%%%%%%   SECTION   %%%%%%%%%%%%
%%%%%%%%%%%%%%%%%%%%%%%%%%%%%%%%%%%%%
%%%%%%%%%%%%%%%%%%%%%%%%%%%%%%%%%%%%%
\section{Contributions}
\label{sec:intro:contrib}


Why is this an important problem ?
%

In summary, the main contributions of this thesis are:

\begin{itemize}
    \item \bf Measurements
    
    \item \bf ...
    
%\item{\bf Tools}
%  \begin{itemize}
%    \item {\it Design and implementation of measurement tool \textit{Moni} to collect information:} In order to be able to analyze the impact, we needed direct measurements from the lower layer.
%  \end{itemize}
%\item {\bf Algorithm \& Implementation}
% 	\begin{itemize}
%    	\item {\it Linux sublayer extension to enable allocation per packet:} To enable setting per packet we extend Linux subsystem.
%    	\item {\it Design and implementation of our joint controller}: We designed and implemented a joint controller within the Linux kernel.
%	\end{itemize}
%\item {\bf Measurements}
%  \begin{itemize}
%      \item {\it Feasibility of parameter control and its constraints:} We explore the capabilities of today's hardware and explored the ability to set different parameters.
%	    \item {\it Validation and Performance Analysis:} We perform several validation experiments that confirm trobust operation.
%  \end{itemize}
\end{itemize}

%%%%%%%%%%%%%%%%%%%%%%%%%%%%%%%%%%%%%
%%%%%%%%%%%%%%%%%%%%%%%%%%%%%%%%%%%%%
%%%%%%%%%%%%   SECTION   %%%%%%%%%%%%
%%%%%%%%%%%%%%%%%%%%%%%%%%%%%%%%%%%%%
%%%%%%%%%%%%%%%%%%%%%%%%%%%%%%%%%%%%%
\section{Thesis outline}
\label{sec:intro:outline}

The rest of this thesis is organised as follows:

\begin{compactitem}
  \item In Chapter 2, we discuss the different algorithms, measurement parameter and medical applications.
  \item In Chapter 3, we present our Testbed in detail.
  \item In Chapter 4, we present our measurement methods and results.
  %\item In Chapter 5, we present the design, the Linux implementation, the validation and performance evaluation of our controller.
  \item In Chapter 5, we summarise the contributions and limitations of our systems, and outline several directions for future work.
\end{compactitem}